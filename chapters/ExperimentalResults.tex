\chapter{Experimental Results}
\label{sec:ExperimentalResults}

In order to evaluate the efficiency of the designed packages and also the Apriltag's detection algorithm, a series of experiments were conducted both in simulation and with the real system.

\section{Experiments Conducted in Simulation}
\label{sec:simulatedExperiments}

In order to evaluate the simulated system, the ability of the MAV to track and follow the Apriltag under various conditions was tested. As mentioned in \ref{sec: apriltagFireflySimulation} there are two offsets added in the robot's reference frame, these target to the security of the user, so as to avoid collision, and also to provide the MAV with a better field of view, thus improve the Apriltag detection rate.   

We prove that the MAV always leaves an almost constant distance between itself and the detected marker. First the user presents the Apriltag that commands the MAV to hover. Then the marker that commands it to follow, after that the program automatically rotates this marker inside the simulation with a constant predefined turning rate. This is accomplished by changing the marker's yaw, while leaving the position unchanged. The tested rates are $0.5^{\circ}$, $2^{\circ}$ and $5^{\circ}$ per frame and since in simulation the USB camera records with 30 fps, the rates are of $15^{\circ}/sec$, $60^{\circ}/sec$ and $150^{\circ}/sec$ respectively. In figure \ref{pics:xyRotation}, only the x and y coordinates are presented. It can be seen that the MAV performs a circle while it follows the tag, thus it leaves a constant distance between itself and the marker. The lines that start from point (0,0) represent the Firefly's trajectory to its initial position relative to the tag. The spikes that appear in the $60^{\circ}/sec$ and $150^{\circ}/sec$ trajectories exist because by the time the MAV reached its initial position the tag had already rotated, thus the robot had to move greater distance. The fastest rotation the MAV can follow is about $240^{\circ}/sec$, at greater rotation speeds the MAV is not able to follow the rotation. Conversely to what was expected, the Firefly has more oscillations in the z axis at the lowest turning rate, as can be seen from figure \ref{pics:zRotationCOntroller}. That is normal since the MAV has to stay at the various positions before it receives the next measurable position value. Thus, the MAV has to stop and continue at all times in order to follow the Apriltag. On the other hand, when the tag is rotated faster the MAV always gets a new position command before it needs to stop and wait for a new one.

\begin{figure}
   \centering
   \includegraphics[width=1.00\textwidth]{images/MAV_controller_input_rotation_tracking.pdf}
   \caption{The user holds the Apriltag still and the algorithm automatically rotates it around z-axis. The MAV follows the rotating Apriltag and keeps a constant distance at all times. The spikes are caused because by the time the MAV reaches the position from the initial detection, the algorithm has already rotated the marker, thus the MAV "jumps" in order to cover that distance.}
   \label{pics:xyRotation}
\end{figure}

\begin{figure}
   \centering
   \includegraphics[width=0.98\textwidth]{images/sim_z_axis_during_rotation.pdf}
   \caption{The controller input with respect to z-axis versus time during the three rotation experiments, that were conducted in simulation. The z position oscillates more at slower rotation speeds, and less at faster rotation speeds. The more oscillations at slower rotation speeds are caused from the fact that the MAV has to halt mid-air in order to wait the Apriltag rotate itself so as to get new detection data.}
   \label{pics:zRotationCOntroller}
\end{figure}

 
\section{Experiments Conducted in the Real System}
\label{sec:realExperiments}

In order to validate our system, experiments with a real MAV were conducted. The name of the drone used is "AUK". First of all, during the experiments, it was observed that when the marker was moved at distances further than 2m away from the camera, then the orientation detection became unreliable and changed randomly. Although the algorithm could still measure distance accurately, it had some significant problems reading the correct orientation data from the Apriltag. That misdetection led to very abrupt random motions, since the coordinates sent to the controller use the detected pose of the marker in the transformation from the reference frame to the world frame. Thus, in order to avoid any dangerous situations, the orientation read from the Apriltag detection algorithm was substituted with a fixed rotation. Due to the aforementioned, the MAV could only follow the marker in straight lines since it did not track the Apriltag's yaw. Furthermore, during the experiments we experienced some difficulties with the onboard odometry system (ROVIO\protect\footnotemark), thus the VICON system was used to provide odometry data.

\footnotetext{https://github.com/ethz-asl/rovio}

Four experiments were performed, in the first three the ability of the MAV to follow a "normally" moving target was tested, while in the fourth experiment the marker motions became faster and more abrupt. It should be mentioned, that during the first experiment we faced some difficulties due to problems with the MAV's trim. The overall Apriltags' detection rates based on the experimental results, are presented in table \ref{table:detctionRates}. In order to compute the detection rate, the number of frames during the manual phases (landing and take off) were excluded from the total number of frames, so for the first experiment we had 2106 frames at which the marker was detected out of 2317 frames, 2020 detections out of 2089 for the second experiment, 786 detections out of 816 for the third and 1122 out of 1205 for the last experiment. It can be seen that the algorithm, regardless the motion of the Apriltag achieved at least 90\% detection at all cases.

\begin{table}
\begin{center}
    \begin{tabular}{ | l | l |}
    \hline
    Experiment & Detection Rate \\ \hline
    $1^{st}$ Experiment & 90.89\% \\ \hline
    $2^{nd}$ Experiment & 96.69\% \\ \hline
    $3^{rd}$ Experiment & 96.32\% \\ \hline
    $4^{th}$ Experiment & 93.11\% \\ \hline
    \end{tabular}
    \caption{In this matrix are presented the Apriltag detection rates, as computed from the experimental data taken from the four conducted experiments. We derive these percentages by comparing the number of frames at which there was a marker detection, to the overall number of frames at which the MAV was not flown by the user. For the first experiment the frames at which we had Apriltag detection were 2106 out of 2317 frames, 2020 out of 2089 frames for the second, 786 out of 816 frames for the third and 1122 out of 1205 for the last conducted experiment.}
    \label{table:detctionRates}
\end{center}
\end{table}

\begin{figure}
	\centering
	 \includegraphics[width=1.00\textwidth]{images/real_MAV_odom_controller_bag3.pdf}
	 \caption{In this figure, we can see the Apriltag as detected from the MAV, the position controller input, which is the position towards which we want the MAV to move and the odometry output, which depicts the actual position of the MAV in space.}
	 \label{pics:realOdomControlDetection}
\end{figure}

\begin{figure}
	\centering
	 \includegraphics[width=1.00\textwidth]{images/real_euclidean_distance.pdf}
	 \caption{This figure depicts the euclidean distance between the MAV and the Apriltag in x axis versus time. This the actual distance between the user and the MAV. As can be seen the most of the time could not achieve the safety distance of 1.5m because the user changed the desired position, by moving the marker, before the MAV could reach it. Nonetheless, the deviation from the predefined distance is on average 11.67cm.}
	 \label{pics:euclideanDistance}
\end{figure}

Since the nature of the experiments is very much alike to each other, only one set of data will be presented. As can be seen from figure \ref{pics:realOdomControlDetection}, the MAV manages to track the marker and follow it under all circumstances. The MAV, acts like a low pass filter on the controller inputs, thus it smoothens the aforementioned as it can be seen from the odometry output. Furthermore, we can observe from figure \ref{pics:euclideanDistance} the euclidean distance between the Auk and the marker with respect to x axis as it proceeds in time. Although the distance between the MAV and the user is, almost at all cases, less than the predefined safety distance, the MAV always leave adequate distance as it can be seen from the orevious figure. The reason for not always achieving the safety distance, is that the user moves the Apriltag before the MAV has reached that predefined distance, but even in this case, the average error from our desired safety distance is only 11.67 cm.   


During the last experiment, the ability of the algorithm to track faster and more abrupt motions was tested. During this experiment, some discontinuities in the MAV's motion were observed, they were caused by the instantaneous loss of the Apriltag detection, since the user moved the marker fast from one position to another. During the time between the last observed marker position and until a new detected position was created from the detection algorithm, the MAV's controller was receiving as target position the last position where the marker was detected. This led to short sudden holds in the MAV's trajectory. During the analysis of the data, it was observed that the periods when the MAV could not detect the Apriltag varied between two to nine frames, or 66 to 300 msec. But since the MAV had already started to move to the direction of the marker, it was able to detect and follow it again.
 

